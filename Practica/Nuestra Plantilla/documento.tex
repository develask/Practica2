\documentclass[a4paper,10pt]{article}
\usepackage[margin=0.5in]{geometry}
\usepackage{amsthm}
%\usepackage{mathtools}
\usepackage{amsfonts}
\usepackage{amssymb}
\usepackage{array}
\usepackage[round]{natbib}
\usepackage{array}
\theoremstyle{plain}
\usepackage{color}
\newtheorem{theo}{Theorem}
\newtheorem{defn}{Definition}

% Redefinimos la funcion \today para ponerla en castellano
\def\today{\number\day~de\space\ifcase\month\or 
  enero\or febrero\or marzo\or abril\or mayo\or junio\or 
  julio\or agosto\or septiembre\or octubre\or noviembre\or diciembre\fi 
  \space de~\number\year}
  
\begin{document}
\author{Mikel de Velasco}
\pagenumbering{roman}
\title{Pr\'actica 2}

\author{Mikel de Velasco, Ieltzu Irazu, M� In\'es Fernandez}
\date{\today}
\maketitle

%\begin{abstract}
%In this manuscript, we study the Boltzmann distribution associated to the linear ordering problem. Particularly, we demonstrate that this distribution is L-decomposable. Moreover, we investigate towards an efficient sampling method of the Boltzmann distribution.
%\end{abstract}


\section{Introducci\'on}

Se pide diseñar un clasificador que implemente el m\'etodo k-NN b\'asico con la distancia de
Minkowski seg\'un la expresi\'on (\ref{minkowski}) donde {\color[rgb]{1,.5,0} n} representa el n\'umero de atributos empleados para caracterizar las muestras.

\begin{equation}\label{minkowski}
d(a,b)=\left[\sum_{i=1}^{{\color[rgb]{1,.5,0} n}}|a_{i}-b_{i}|^{\color{magenta} m}\right]^{\frac{1}{\color{magenta} m}}
\end{equation}

El clasificador permitir\'a seleccionar tanto el n\'umero de vecinos a explorar ({\color{blue} k}) como el par\'ametro {\color{magenta} m} de la expresi\'on (\ref{minkowski}). Elegir el lenguaje de programaci\'on que se considere m\'as apropiado para el diseño.
Para inferir el clasificador se dispone de un conjunto de datos (Diabetes.arff). El conjunto de datos dispone de 768 instancias para inferir el modelo. Para describir las instancias se utilizan 8 atributos m\'as la clase. Los atributos son los siguientes:

\begin{itemize}
\item Number of times pregnant
\item Plasma glucose concentration a 2 hours in an oral glucose tolerance test
\item Diastolic blood pressure 
\item Triceps skin fold thickness 
\item 2-Hour serum insulin 
\item Body mass index 
\item Diabetes pedigree function
\item Age 
\item Class variable 
\end{itemize}

La clase que hay que determinar es si el paciente tiene diabetes o no.
    
\section{Metodolog\'ia}

En este apartado desarrollaremos nuestro propio algoritmo K-Nearest Neighbors y describiremos el diseño y como hemos implementado este algoritmo. Adem\'as haremos una breve descripci\'on de como hacer funcionar el programa.


\section{Resultados}

\section{Conclusiones}

\section{Valoraci\on Subjetiva}
\subsection{Par\'ametro {\color{blue}k}}
\subsection{Par\'ametro {\color{magenta}m}}

\section*{Acknowledgements}
Thank you every one!

\bibliographystyle{plainnat}
\bibliography{./bibliography.bib}

\end{document}

